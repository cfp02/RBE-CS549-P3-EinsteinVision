\section{Reflection}

% Our team put in many hours of work into this project, and most of that time was spent attempting to get techniques described in literature to run on our data, or even run at all. We found many powerful techniques, all with source code available, but despite this could not manage to get things actually working in time for the deadlines. This was in part due to time constraints within the group, but also due to the nebulous nature of the project. In previous projects, the goals were very clear, and the techniques could be \emph{fully} understood by our team, and thus knew when we were on the right track or not. With this project in contrast, our team found it hard to judge whether pursuing any given technique was worth our time, and thus found it hard to make progress, as we wanted to avoid committing to a technique which would not work.

% As a result of this, while our team spent considerable time working on this project, it felt like the only way of getting better results was simply dedicating more time to the tedious task of debugging outdated python dependencies. 

% Our team also experienced a lot of fatigue when working on the project, especially in comparison to the previous projects in this course, as we did not feel as though we were learning anything or making progress as we spent dozens of hours struggling to perform inference using techniques we discovered. We didn't feel we had the time implement our own techniques, but also felt as though we were not learning anything by trying to get other people's techniques to work. Since this goal of this project was focused on "good" output, it made it stressful to prioritize different tasks, short of simply putting more time into everything, as there was no guarantee that any given pathway was worthwhile. This was a very frustrating experience for our team. This in turn made it harder to justify taking more than dozens of hours each week we spent on the project, as it felt as though we could have spent many more dozens of hours without making any progress, or learning anything new.

Our team put in many hours of work into this project, and most of that time was spent attempting to get techniques described in literature to run on our data, or even run at all. We found many powerful techniques, all with source code available, but despite this could not manage to get things actually working in time for the deadlines. We noted that the task of reproducing results from pervious papers, and generalizing the code to new data was tedious, and also very exhausting, frustrating, and time consuming. This made it difficult to make progress, as we wanted to avoid committing to a technique which would not work. The unbounded nature of the target goals for this project also made it harder to keep work focused and directed at the most important tasks. To address these challenges, our team frequently created priority lists and split up work into tasks which were entirely parallelizable to ensure that the most worthwhile tasks were completed first, and that work was never held up by the progress of the other individual.

The second aspect of this project which made it difficult to motivate (but something which cannot be avoided with this kind of work) was the fact that the project was not about \emph{understanding} or developing intuition or rigorous knowledge about a topic, but rather was concerned purely with results. This project showed the difficulties with the latter goal, as well as allowed our team to gain a first hand idea of what we enjoyed and did not enjoy about this aspect of the field of computer vision.

On the positive side, while we may not have found the actual work enjoyable, and did not learned much about the actual techniques being used, as it would have simply been too much work, we did learn about the \emph{process} of doing reviews of papers and replicating results. We learned both what makes it challenging and frustrating, and also the value in having experience in doing so. This project gave our team a initial taste of this experience, and also provided a good opportunity to learn about the challenges of working with other people's code, and the importance of good documentation and code quality. The challenges we faced in this project provided motivation to continue to ensure that the code we write is reproducible, understandable, and applicable by others.